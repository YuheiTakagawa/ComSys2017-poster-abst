\documentclass{ipsjpapers}
\usepackage[dvipdfmx]{graphicx}
\usepackage{comsys-poster}
\usepackage{multirow}
\usepackage{url}
\title{\centerline{異種OSコンテナ・マイグレーション実現に向けた}\\Linux-FreeBSD間プロセス・マイグレーション実現方式の検討}

\affilabel{FUN}{公立はこだて未来大学}
\author{高川 雄平\affiref{FUN} \and 松原 克弥\affiref{FUN}}
\renewcommand{\baselinestretch}{0.97}
\begin{document}
\maketitle
\section{はじめに}
クラウドコンピューティング分野において,コンテナ型仮想化が注目されてい
る\cite{focus-container}.コンテナ型仮想化は,OSが提供する資源を分離・制限し,単一動作
するOS上に複数の独立した実行環境を構築できる軽量な仮想化を実現する.
また,クラウドコンピューティング実働環境では,負荷分散や可用性の実現を目的としてライブマイグレーションが活用されている.ライブマイグレーションとは,サービスを実行している仮想マシンを動的に別のマシンに移動させる技術である.コンテナ型仮想化のライブマイグレーション(以下,コンテナ・マイグレーション)は,CRIU(Checkpoint/Restore in Userspace)\cite{criu}やFreeBSD VPS(Virtual Private System)\cite{freebsd-vps}
により実現されている.しかし,コンテナ型仮想化の実装がOSに依存するため,異種OS間におけるコンテナ・マイグレーションは実現されていない.

コンテナ・マイグレーションを行うには,プロセス実行状態のマイグレーション(以下,プロセス・マイグレーション)とプロセス隔離状態のマイグレーションが必要になる.
本稿では,異種OSコンテナ・マイグレーション実現に向けたLinuxとFreeBSD間のプロセス・マイグレーションの実現方式を検討する.

\section{各OSにおけるABIの差異}
\label{se:abi}
一般的に,OSが異なると,準拠するABI(Application Binary Interface)が異なる.ABIとは,アプリケーションとOS間のインタフェース仕様のことで,システムコールやデータ型,データ配置などを指す.実際,LinuxとFreeBSDでは,システムコールとメモリレイアウトに差異がある.

\begin{table}[t]
  \caption{openシステムコールにおけるOSの差異\label{tb:syscall}}
  \vspace{0.5em}
  \begin{center}
    \begin{tabular}{|c|c|c|} \hline
       & Linux & FreeBSD  \\ \hline \hline
      \shortstack{open()の\\システムコール番号} & \raisebox{0.5em}{2} & \raisebox{0.5em}{5} \\ \hline
      \shortstack{open()の\\引数オプション\\O\_CREAT} & \raisebox{1em}{0x0200} & \raisebox{1em}{0x0040} \\ \hline
      引数の渡し方 & レジスタ経由 & スタック経由  \\ \hline
    \end{tabular}
  \end{center}
  \vspace{-0.3cm}
\end{table}

\subsection{システムコール}
UNIX系OSであるLinuxとFreeBSDでは,多くのシステムコールで提供する機能が類似している.しかし,同じ機能のシステムコールにおいても,システムコール番号や引数パラメータ,引数の渡し方が異なる(表\ref{tb:syscall}).そのため,スタックの状態やレジスタの状態が異なり,結果として,LinuxバイナリとFreeBSDバイナリはそれぞれのOS上でしか動作しない.
\subsection{メモリレイアウト}
\label{memory-diff}
メモリレイアウトは,仮想メモリ空間におけるデータ領域やスタック領域などの配置である.FreeBSD以外の多くのOSでは,ASLR(Address Space Layout Randomization)によって,テキスト,データ,ヒープ,スタックの各領域がランダムに配置される.また,ASLRを無効化して領域の配置を固定した場合でも,LinuxとFreeBSDでは,ヒープ領域,スタック領域,および,共有ライブラリの配置に差異がある.この差異を補正するためには,スタック内にあるベースアドレス,リターンアドレス,共有ライブラリ内の命令を示すアドレス,ヒープ領域の変数を参照するアドレスを全て特定して値を書き換える必要が生じる.

\section{プロセス・マイグレーションの実装}
プロセス・マイグレーションは,対象プロセスの実行状態を示すレジスタやメモリの状態を保存(スナップショット)し,別のOS環境で生成したプロセスに状態を復元(レストア)することにより実現できる.

\subsection{スナップショット}
対象プロセスに関するレジスタとメモリの情報の取得方法について述べる.

レジスタ情報は,ptraceシステムコールによるプロセスへのアタッチ時に,GETREGSコマンドを用いて,レジスタ情報を格納する構造体(以下,レジスタ構造体)から取得する.レジスタ構造体には,汎用レジスタ,インデックスレジスタ,セグメントレジスタ,フラグレジスタの各値が格納されている.

メモリ情報に関しては,ptraceシステムコールのアタッチ時に,procfsのmemファイルを読み込むことで取得できる.ただし,\ref{se:conv}節で述べるLinuxバイナリ互換機能を用いることで,LinuxとFreeBSDで同じLinuxバイナリを実行できるため,テキスト領域と共有ライブラリは保存する必要がない.

\subsection{ABI変換}
\label{se:conv}
\ref{se:abi}章で述べたように,OS毎にABIが異なる.本実装では,ABIの変換により各OSのプロセス実行状態の表現差異を吸収する.

システムコールの差異は,FreeBSDカーネルのLinuxバイナリ互換機能\cite{linux-emu}を用いて,システムコール番号,引数値の変換,引数の渡し方の相互互換性を実現する.Linuxバイナリ互換機能によって,LinuxとFreeBSDの対象プロセス実行時におけるスタックとレジスタの状態を同じ表現にすることができる.

メモリレイアウトの差異に関して,Linuxでは,prctlシステムコールのメモリレイアウト変更機能を用いる.メモリレイアウト変更機能は,プログラムやライブラリのロード時に決まるメモリレイアウトを,プロセス実行中に変更できる.スナップショット時とレストア時のメモリレイアウトを同じレイアウトにすることで,メモリ内部を変更することなく,メモリレイアウトの差異に対応させる.メモリレイアウト変更機能を用いることで,ASLRによるプロセス生成毎のランダムなメモリレイアウトにも対応できる.しかし,FreeBSDは,Linuxのメモリレイアウト変更機能に相当する機能がないため,新たに実装する必要がある.

\subsection{レストア}
レストアは,スナップショットで作成したファイルを用いて,転送先で生成したプロセスに対して,レジスタとメモリの復元を行う.プロセスの実行状態を復元するためには,Cプログラムにおけるmain関数が実行されるまでにレジスタとメモリの状態を復元する必要がある.ELFヘッダで取得できるエントリーポイントの呼び出しの直前に,レジスタやメモリの状態を復元することを検討したが,本手法で用いるLinuxバイナリ互換機能がmain関数実行前の初期化処理内で有効化されるため,エントリーポイント呼び出し時の実行状態復元は適切でない.本手法では,main関数の最初の命令をint3命令\footnote{デバッガへのトラップ用割り込みを発生させるアセンブラ命令}に置き換えることで,実行状態復元の機会を再現する.

レジスタの復元は,int3命令による停止時に,ptraceシステムコールのSETREGSコマンドにより行う.Intel x86アーキテクチャでは,レジスタ構造体の情報を復元する際に,構造体の差異とセグメントレジスタの扱いの差異を考慮しなければならない.Linuxでは,システムコール番号をメンバ変数orig\_raxに格納し,システムコールの返り値をメンバ変数raxに格納する.FreeBSDでは,どちらもメンバ変数raxに格納する.ripメンバ変数が指す命令からシステムコール実行の前後を判断して,FreeBSDのraxに格納する値を決める必要がある.また,メンバ変数の順番も異なるため,各OSが持つ構造体のメンバ変数に合わせて格納しなければならない.セグメントレジスタは,OSがプロセス生成時に割り当てた値を使用する必要がある.本手法では,スナップショット時に取得したセグメントレジスタの値を使用せず,復元先のOSが割り当てた値をそのまま利用する.

メモリの復元は,int3命令による停止時に,procfsのmemファイルを介して,データ,ヒープ,スタックの各領域のメモリ状態を書き込む.この際,\ref{se:conv}節で述べたメモリレイアウト変更機能を用いてスナップショット時のメモリ配置を再現することで,メモリ情報の内容を変更せずにプロセスを復元できる.
\section{おわりに}
本稿では,異種OS間コンテナ・マイグレーション実現に向けた,Linux-FreeBSD間プロセス・マイグレーション実現方式の検討を行った.システムコールに関する差異はLinuxバイナリ互換機能を用い,メモリレイアウトに関する差異はメモリレイアウト変更機能を用いて吸収する.\\
 今後の課題として,実運用に向けてネットワークソケットやファイルポインタなどのカーネル等プロセス外で管理されているプロセス関連情報を対象プロセスのマイグレーションに含める技術の実現がある.
\begin{thebibliography}{99}
  \bibitem{focus-container}
  	451 Research, "451 Research: Application containers will be a \$2.7bn market by 2020", 2017.
  \bibitem{criu}
  	CRIU Project: CRIU Main page, \url{https://criu.org/}, 2010 (accessed November 1 2017).
  \bibitem{freebsd-vps}
  Klaus P. Ohrhallinger: Virtual Private System for FreeBSD, \url{http://www.7he.at/freebsd/vps/}, 2010 (accessed November 6, 2017). EuroBSDCon 2010.
  \bibitem{linux-emu}
  Jim Mock: Linux Binary Compatibility, https://www.freebsd.org\slash{}doc\slash{}handbook\slash{}\\linuxemu.html, 1999 (accessed November 2, 2017).
\end{thebibliography}
\end{document}
