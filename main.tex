\documentclass{ipsjpapers}
\usepackage[dvipdfmx]{graphicx}
\usepackage{comsys-poster}
\usepackage{multirow}
\usepackage{url}
\title{異種OS混在環境におけるプロセス・マイグレーションの実現}

\author{高川 雄平 松原 克弥}

\begin{document}
\maketitle
\section{はじめに}
近年,クラウドコンピューティングの基盤技術の一つであるコンテナ型仮想化が注目されてい
る\cite{focus-container}.コンテナ型仮想化は,OSが提供する資源を分離・制限し,単一動作
のOS上に複数の独立した実行環境を構築できる軽量な仮想化を実現する.\\
 また,クラウドコンピューティング実動環境では,負荷分散や可用性の実現を目的にライブマイグ
レーションが活用される.ライブマイグレーションとは,サービスを実行している仮想マシンを動的に
別のマシンに移動させる技術である.コンテナ型仮想化におけるライブマイグレーションは,Linux
ではCRIU (Checkpoint/Restore in Userspace)\cite{criu},FreeBSDではFreeBSD VPS (Virtual Private System)\cite{freebsd-vps}
が実現している.しかし,コンテナ型仮想化の実装はOS依存のため,異種OS混在環境間でのコンテ
ナ型仮想化のライブマイグレーションは実現されていない.\\
 コンテナ型仮想化実行環境のマイグレーションを行うには,プロセス実行状態のマイグレーション(プロセス・マイグレーション)とプロセス隔離状態のマイグレーション(コンテナ・マイグレーション)が必要になる.
\\
 本稿では,異種OS混在環境間におけるコンテナ型仮想化実行環境のライブマイグレーション実現に向けた、異種OS間のプロセス・マイグレーションの実現方式の検討と実装を行う.

\section{異種OS間プロセス・マイグレーション実現方式}
本章では,異種OS間でのプロセス・マイグレーションの実現方式を検討
する.本稿の対象OSは,LinuxとFreeBSDとする.\\
異種OS混在環境におけるプロセスマイグレーションの実現には,以下の課題を解決する必要がある.

\begin{table}[t]
  \label{tb:syscall}
  \caption{openシステムコールにおけるOSの差異}
  \vspace{0.5em}
  \begin{center}
    \begin{tabular}{|c|c|c|} \hline
       & Linux & FreeBSD  \\ \hline \hline
      \shortstack{open()の\\システムコール番号} & \raisebox{0.5em}{2} & \raisebox{0.5em}{5} \\ \hline
      \shortstack{open()の\\引数オプション\\O\_CREAT} & \raisebox{1em}{0x0200} & \raisebox{1em}{0x0040} \\ \hline
      引数の渡し方 & レジスタ経由 & スタック経由  \\ \hline
    \end{tabular}
  \end{center}
\end{table}

\subsection{システムコールに関する差異}
\label{se:syscall}
LinuxとFreeBSDでは,システムコールの実装は大半が同じである.しかし,システムコール番号や引数パラメータ,引数の渡し方が異なるため,LinuxバイナリとFreeBSDバイナリは各OS上での
みしか動作しない.表\ref{tb:syscall}には,openシステムコールにおける差異を示している
.openシステムコールの番号は,Linuxでは2,FreeBSDでは5となっている.openシステムコールでファ
イルを作るときのパラメータとして渡すO\_CREATは,Linuxでは0x0200,FreeBSDでは0x0040となって
いる.システムコールの引数を,Linuxではレジスタ経由,FreeBSDではスタック経由で渡す
.以上のようなシステムコールに関する差異があるため,メモリの状態やレジスタの状態が異なる
.\\
 システムコール番号,引数パラメータの変換を行い,引数の渡し方を互換することで解決すること
ができる.FreeBSDにはLinuxバイナリ互換機能\cite{linux-emu}というカーネル機能があり,上述の変換と互換
を行う.この機能によって,Linuxバイナリであれば,Linux
とFreeBSDでバイナリファイルを一切変更せずに動作させることができる.本稿では,システムコールの差異をLinuxバイナリ互換機能を用いることで解決する.

\subsection{メモリレイアウトに関する差異}
\label{se:memory}
メモリレイアウトは仮想メモリ空間における領域の配置である.一般的にASLR (Address Space
Layout Randomization)によって,領域はランダムに配置される.また,ASLRを
無効化して領域を固定した場合でも,LinuxとFreeBSDではヒープ領域・スタック領域・共有ライブラリの配置に誤差がある.この誤差を純粋に修正する場合には,スタック内にある全てのベースアドレス,リターンアドレス,共有ライブラリ内の命令を示すアドレス,ヒープ領域の変数を参照するアドレスなどを見つけ出し配置の誤差だけ移動する必要がある.\\
 毎回全ての領域に対して,メモリ内部の探索と移動を行うのはオーバーヘッドが大きい.本稿では,メモリレイアウト変更機能を利用し,マイグレーション元とマイグレーション先のメモリレイアウトを同じレイアウトに設定することで,メモリ内部を変更することなく解決する.同じレイアウトにすることができるため,ASLRのランダムなメモリレイアウトも再現可能である.\\
 Linuxでは,prctlシステムコール機能を利用することでメモリレイアウトをユーザプログラムから変更可能である.しかし,FreeBSDにはメモリレイアウトをユーザプログラムから変更できる機能はないため,実装する必要がある.

\section{プロセス・マイグレーションの実装}
プロセス・マイグレーションを行うには,プロセスの実体であるレジスタやメモリの状態を保存(チェックポイント)し復元(レストア)する必要がある.プロセスの監視には,ptraceシステムコールを用い,監視する親プロセス(tracer)が監視される子プロセス(tracee)にレジスタやメモリ情報の取得・復元のための制御を行う.なお,FreeBSDのLinuxバイナリ互換機能は,エントリーポイントからmain関数の呼び出しまでのスタートアップと呼ばれる処理で利用されるため,エントリーポイントではなく,main関数の呼び出し時にレストア処理を行う必要がある.

\subsection{メモリ}
チェックポイント時にメモリの情報は,procfs内にあるtraceeプロセスのmemファイルをtracerから読み取ることで取得できる.第\ref{se:syscall}章で述べたように,Linuxバイナリを使うことで,メモリ内部の状態は環境変数以外が同じ状態になる.第\ref{se:memory}章のように,メモリレイアウトを同じ配置にすることで,メモリ内部の値を変更せずに,異種OSであってもメモリ状態を同じにすることができる.レストア時にはprocfsのtraceeプロセスのmemファイルを開き,データ領域,ヒープ領域,スタック領域のそれぞれ対してシークを行い,チェックポイント時のメモリファイルを書き込むことで復元する.また,テキスト領域や共有ライブラリは,同じLinuxバイナリを用いるため,復元する必要がない.

\subsection{レジスタ}
チェックポイント時にレジスタの情報を取得するには,ptraceシステムコールのGETREGSを用いて,レジスタ情報を格納する構造体から取得できる.Linux\slash{}x64とFreeBSD\slash{}x64では,GETREGSで取得できる情報は汎用レジスタ,インデックスレジスタ,特殊レジスタ,セグメントレジスタなどが取得できる.\\
 レジスタを復元するには,レジスタ情報を格納する構造体の差異を考慮しなければいけない
.LinuxとFreeBSDとでは,メンバ変数の宣言の順番が異なるため,チェックポイント時の構造体をそ
のまま利用して復元することができない.また,Linuxでは呼び出すシステムコール番号をメンバ変
数orig\_raxに格納し,システムコールの返り値をメンバ変数raxに格納するが,FreeBSDではどちらも
メンバ変数raxに格納する.システムコール呼び出し前に限定し,チェックポイント時のレジスタ値を各OSが持つ構造体のメンバ変数に合わせて格納することで,正しく復元することができる.\\
 また,セグメントレジスタはLinux\slash{}x64とFreeBSD\slash{}x64では,各OSがプロセスごとに
値を割り当てるため,OSに割り当てられた値以外を利用すると,セグメントフォルトが起こる.その
ため,セグメントレジスタはチェックポイント時の値を利用せず,OSに割り当てられた値をそのまま
利用する.
\section{まとめ}
本稿では,異種OS混在環境におけるコンテナ型仮想化のライブマイグレーション実現に向けた,プロセス・マイグレーションの検討と実装を行った.対象OSをLinuxとFreeBSDとし,システムコールに関する差異はLinuxバイナリ互換機能を用い,メモリレイアウトに関する差異は同じメモリレイアウト機能を用いて吸収する.\\
 今後の課題として,実運用に向けてネットワークソケットやファイルポインタなどのカーネル等プロセス外で管理されているプロセス関連情報を対象プロセスのマイグレーションに含める技術の実現がある.
\begin{thebibliography}{99}

  \bibitem{focus-container}
  	451 Research, "451 Research: Application containers will be a \$2.7bn market by 2020", 2017.
  \bibitem{criu}
  	CRIU Project: CRIU Main page, \url{https://criu.org/}, 2010 (accessed November 1 2017).
  \bibitem{freebsd-vps}
  Klaus P. Ohrhallinger: Virtual Private System for FreeBSD, \url{http://www.7he.at/freebsd/vps/}, 2010 (accessed November 6, 2017). EuroBSDCon 2010.
  \bibitem{linux-emu}
  Jim Mock: Linux Binary Compatibility, https://www.freebsd.org\slash{}doc\slash{}handbook\slash{}\\linuxemu.html, 1999 (accessed November 2, 2017).
\end{thebibliography}
\end{document}
